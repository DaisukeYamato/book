% 1
%\documentclass[comp]{icu} % 提出用のフォーマット
\documentclass{jsarticle}
\usepackage[dvips]{graphicx}
\usepackage{verbatim}

%==== タイトル部分
%\title{タイトル}
%\題名{日本語の題名}
%\author{SUZUKI, Nanigashi}
%\氏名{鈴木 某}
%\date{March, 2002} % 自分の卒業する時期は自分で書き込もう
%==== タイトル定義ここまで

\begin{document}
%\maketitle % これで中表紙ができる
%\frontmatter
%\tableofcontents\newpage
%\listoftables\listoffigures
%\input{acknowledgment}
%\mainmatter
\setlength{\baselineskip}{2em} % 行間の自動調整を停止したので
                               % 本文の直前にこの1行を加えて対処をお願いします
\setcounter{section}{0}
\section{イントロダクション}
この読み物は、"An Introduction to Statistical Learning"(http://www-bcf.usc.edu/~gareth/ISL/ から入手可能。)
の解説として大和(以降、筆者とする)が記したものである。
はじめに断っておくが、これは単なる日本語に訳したものではない。
重要な部分をピックアップし、要約・加筆・修正したものである。
ただし、図などは(今のところ)貼付ける予定はないので、
Figureとかこの文章中に出くわせば、適宜英語の図の所を参照してほしい。
(本当はこのテキストに貼付ければいいのだろうが、めんどくさい・・・)

機械学習は、ざっと言えば何らかのデータを理解するための一つのツール(もしくは
複数の機械学習を用いるならばツール群と言った方がいいかもしれない)である。
機械学習は、大別すると以下の2つに分類される。
\begin{itemize}
\item 教師付き学習 (supervised):1つ、あるいは複数入力。アウトプットは1つ。
\item 教師なし学習 (unsupervised):1つ、あるいは複数入力。アウトプットはない。データ構造や関係性を探ることを目的とする。
\end{itemize}

以下、この本で使用されるデータについて述べる。
ただし、いずれのデータおよび分析手法は後のチャプターで詳しく述べられるはずなので、
今は概観をつかんでもらえればいいと思う。

\subsection{給与データ}
 想像できるように、労働者の給与は社会人経験年数に関係している。
また、教育にも依存するだろう(大学卒業、大学院卒業など)。
後の章で、回帰分析の際このデータを使用する(データに関しては、Figure 1.1参照)。

\subsection{株式市場データ}
 株価自体は、ある正の値を持っていて、将来の株価予想などが行えれば便利である。
ただ現実は予想が難しい(予想できれば、こんな訳してる場合じゃなく株取引を積極的にしてるはず)。
一歩条件を緩くして、価格を正確に予測する代わりに、前日から今日の価格が上がるか、もしくは
下がるかだけでも予測できないだろうか?(ということで、そんなデータの説明がFigure 1.2。)
ただ、こういう分析は教師付き学習ではあるけれでも、離散的なアウトプットをするので
(前日に比べて価格が上がる、もしくは下がるの2通り)、統計では分類(Classification)と呼ばれる(回帰に対して)。

株価予想モデルは、チャプター4で行うが、60\%の確率で上下変動を予想できるみたい。

\subsection{遺伝子情報データ}
 教師なし学習として、遺伝子情報データを最後に紹介する。
インプットデータは遺伝子の並び方であるが、アウトプットとして
何か値を出すのではなく、インプットデータの特徴を捉えることを目的としている。
(Figure 1.4に遺伝子情報を何らかのスコア2つ$Z_1$, $Z_2$に焼き直して
クラスターした結果が示されている。
左の図は機械学習による結果で、右はガンの種類14種類を表したもの。
この場合、ガンの種類の情報はデータとして持っているので、
分類と捉えることも出来そう。)

\subsection{機械学習の歴史}
割愛。興味がある人はどうぞ。
ただし、この本"An Introduction to Stastical Learning"のベースとなる本があって
"The Elements of Statistical Learning" (http://statweb.stanford.edu/~tibs/ElemStatLearn/ から入手可能)、
この本が終われば読んでみるのもいいかも。(筆者は大学院のときにBoosted Decision Treesの部分だけ読みました)

\subsection{記法}
割愛しますが、この本は極力行列表現をなくしているみたいです。

\subsection{データセットなどRに関して}
Rでのコードもついています。使用するデータセットは下のライブラリ2つについています b7。
\begin{itemize}
\item ISLR
\item MASS
\end{itemize}
なので、install.packages()でインストールしておいてください。
%=======================================
\end{document}


